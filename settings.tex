%Dokumentenklasse "scrbook" - Erweitert um den Verweis auf die Verzeichnisse und Texteigenschaften
\documentclass[chapterprefix=true, 12pt, a4paper, oneside, parskip=half, listof=totoc, bibliography=totoc, numbers=noendperiod]{scrbook}

% Ränder (Standard bottom ca. 52mm anbzüglich von ca. 4mm für die nach oben rechts gewanderte Seitenzahl)
%Anpassung der Seitenränder
\usepackage[bottom=48mm,left=25mm,right=25mm]{geometry}

% Ränder bei Bedarf zeigen
%\usepackage{showframe}

%Tweaks für scrbook
\usepackage{scrhack}

%Blindtext
\usepackage{blindtext}

%Erlaubt unteranderem Umbrücke captions
\usepackage{caption}

%Stichwortverzeichnis
\usepackage{imakeidx}

%Kompakte Listen
\usepackage{paralist}

%Zitate besser formatieren und darstellen
\usepackage{epigraph}

%Glossar, Stichworverzeichnis
\usepackage[toc, acronym]{glossaries} % Akronyme werden als eigene Liste aufgeführt

%Anpassung von Kopf- und Fußzeile
%beinflusst die erste Seite des Kapitels
\usepackage[automark,headsepline]{scrlayer-scrpage}
\automark{chapter}
\ihead{\leftmark}
\chead{}
\ohead{\thepage}
\ifoot*{}
\cfoot[\thepage]{}
\cfoot*{}
\ofoot*{}
\pagestyle{scrheadings}

%Auskommentieren für die Verkleinerung des vertikalen Abstandes eines neuen Kapitels
%\renewcommand*{\chapterheadstartvskip}{\vspace*{.25\baselineskip}}

%Zeilenabstand 1,5
\usepackage[onehalfspacing]{setspace}

%Verbesserte Darstellung der Buchstaben zueinander
\usepackage[stretch=10]{microtype}

%Deutsche Bezeichnungen für angezeigte Namen (z.B. Innhaltsverzeichnis etc.)
\usepackage[ngerman]{babel}

%Unterstützung von Umlauten und anderen Sonderzeichen (UTF-8)
\usepackage{lmodern}
\usepackage[utf8]{luainputenc}
\usepackage[T1]{fontenc}

%Einfachere Zitate
\usepackage{epigraph}

%Verwendung von Akronymen
\usepackage[printonlyused]{acronym}

%Unterstützung der H positionierung (keine automatische Verschiebung eingefügter Elemente)
\usepackage{float} 

%Erlaubt Umbrüche innerhalb von Tabellen
\usepackage{tabularx}

%Erlaubt Seitenumbrüche mit Tabellen
\usepackage{longtable}

%Erlaubt die Darstellung von Sourcecode mit Highlighting
\usepackage{listings}

%Definierung eigener Farben bei nutzung eines selbst vergebene Namens
\usepackage[table,xcdraw]{xcolor}

%Vektorgrafiken
\usepackage{tikz}

%Grafiken (wie jpg, png, etc.)
\usepackage{graphicx}

%Grafiken von Text umlaufen lassen
\usepackage{wrapfig}

%Ermöglicht Verknüpfungen innerhalb des Dokumentes (e.g. for PDF), Links werden durch "hidelink" nicht explizit hervorgehoben
\usepackage[hidelinks,german]{hyperref}

%Einbindung und Verwaltung von Literaturverzeichnissen
\usepackage{csquotes} %wird von biber benötigt
\usepackage[style=alphabetic, backend=biber, bibencoding=ascii]{biblatex}
%\addbibresource{references/references.bib}
\addbibresource{references/literatur.bib}

%-------------------------------Zusätzliche Anpassungen und Modifikationen--------------------------------------------%

%Anpassung der Überschriften
\addtokomafont{disposition}{\rmfamily}

%Zusätzliche Farben
\definecolor{darkgreen}{RGB}{0,100,0}

%Umbenennungen
\renewcommand{\lstlistlistingname}{Quelltextverzeichnis}

%Pluszeichen in der Referenc beim zitieren ausblenden
\renewcommand*{\labelalphaothers}{}

%Anpassugen zur Quelltextdarstellung, kann bei Bedarf überschrieben werden (z.B. wenn unterschiedliche Sprachen zum Einsatz kommen)
\renewcommand{\lstlistingname}{Codeauszug}
\definecolor{mygreen}{rgb}{0,0.6,0}
\definecolor{mygray}{rgb}{0.5,0.5,0.5}
\definecolor{mymauve}{rgb}{0.58,0,0.82}
\definecolor{listgray}{rgb}{0.88,0.88,0.88}
\definecolor{sh_comment}{rgb}{0.12, 0.38, 0.18 } %adjusted, in Eclipse: {0.25, 0.42, 0.30 } = #3F6A4D
\definecolor{sh_keyword}{rgb}{0.37, 0.08, 0.25}  % #5F1441
\definecolor{sh_string}{rgb}{0.06, 0.10, 0.98} % #101AF9

\lstdefinelanguage{Javascript}{
  keywords={break, case, catch, continue, debugger, default, delete, do, else, false, finally, for, function, if, in, instanceof, new, null, return, switch, this, throw, true, try, typeof, var, void, while, with},
  morecomment=[l]{//},
  morecomment=[s]{/*}{*/},
  morestring=[b]',
  morestring=[b]",
  ndkeywords={class, export, boolean, throw, implements, import, this},
  keywordstyle=\color{blue}\bfseries,
  ndkeywordstyle=\color{darkgray}\bfseries,
  identifierstyle=\color{black},
  commentstyle=\color{purple}\ttfamily,
  stringstyle=\color{red}\ttfamily,
  sensitive=true
}

\lstdefinelanguage{CSS}{
  morekeywords={accelerator,azimuth,background,background-attachment,
    background-color,background-image,background-position,
    background-position-x,background-position-y,background-repeat,
    behavior,border,border-bottom,border-bottom-color,
    border-bottom-style,border-bottom-width,border-collapse,
    border-color,border-left,border-left-color,border-left-style,
    border-left-width,border-right,border-right-color,
    border-right-style,border-right-width,border-spacing,
    border-style,border-top,border-top-color,border-top-style,
    border-top-width,border-width,bottom,caption-side,clear,
    clip,color,content,counter-increment,counter-reset,cue,
    cue-after,cue-before,cursor,direction,display,elevation,
    empty-cells,filter,float,font,font-family,font-size,
    font-size-adjust,font-stretch,font-style,font-variant,
    font-weight,height,ime-mode,include-source,
    layer-background-color,layer-background-image,layout-flow,
    layout-grid,layout-grid-char,layout-grid-char-spacing,
    layout-grid-line,layout-grid-mode,layout-grid-type,left,
    letter-spacing,line-break,line-height,list-style,
    list-style-image,list-style-position,list-style-type,margin,
    margin-bottom,margin-left,margin-right,margin-top,
    marker-offset,marks,max-height,max-width,min-height,
    min-width,-moz-binding,-moz-border-radius,
    -moz-border-radius-topleft,-moz-border-radius-topright,
    -moz-border-radius-bottomright,-moz-border-radius-bottomleft,
    -moz-border-top-colors,-moz-border-right-colors,
    -moz-border-bottom-colors,-moz-border-left-colors,-moz-opacity,
    -moz-outline,-moz-outline-color,-moz-outline-style,
    -moz-outline-width,-moz-user-focus,-moz-user-input,
    -moz-user-modify,-moz-user-select,orphans,outline,
    outline-color,outline-style,outline-width,overflow,
    overflow-X,overflow-Y,padding,padding-bottom,padding-left,
    padding-right,padding-top,page,page-break-after,
    page-break-before,page-break-inside,pause,pause-after,
    pause-before,pitch,pitch-range,play-during,position,quotes,
    -replace,richness,right,ruby-align,ruby-overhang,
    ruby-position,-set-link-source,size,speak,speak-header,
    speak-numeral,speak-punctuation,speech-rate,stress,
    scrollbar-arrow-color,scrollbar-base-color,
    scrollbar-dark-shadow-color,scrollbar-face-color,
    scrollbar-highlight-color,scrollbar-shadow-color,
    scrollbar-3d-light-color,scrollbar-track-color,table-layout,
    text-align,text-align-last,text-decoration,text-indent,
    text-justify,text-overflow,text-shadow,text-transform,
    text-autospace,text-kashida-space,text-underline-position,top,
    unicode-bidi,-use-link-source,vertical-align,visibility,
    voice-family,volume,white-space,widows,width,word-break,
    word-spacing,word-wrap,writing-mode,z-index,zoom},
  morestring=[s]{:}{;},
  sensitive,
  morecomment=[s]{/*}{*/}
}

\lstset{
  language=Java,
  frame=tb,
  aboveskip=3mm,
  belowskip=3mm,
  frame=tbrl, %t: top, r, b, l 
  frameround=tttt, 
  showstringspaces=false,
  captionpos=b,                    % sets the caption-position to bottom
  columns=flexible,
  basicstyle={\small\ttfamily},
  numbers=left,
  numberstyle=\tiny\color{gray},
  numberblanklines=false,
  %keywordstyle=\color{blue},
  %commentstyle=\color{mygreen},
  %stringstyle=\color{mymauve},
  stringstyle=\color{sh_string},
  keywordstyle = \color{sh_keyword}\bfseries,
  commentstyle=\color{sh_comment}\itshape,
  breaklines=true,
  breakatwhitespace=true,
  xleftmargin=20pt, 
  xrightmargin=10pt, 
  framexleftmargin=15pt, 
  framexrightmargin=5pt,
 % frame=shadowbox,
 % rulesepcolor=\color{gray},
  xleftmargin=20pt,
  escapeinside={<@}{@>}
}
  {}
\lstnewenvironment{Javascript}
  {\lstset{
   language=JavaScript,
   extendedchars=true,
   showstringspaces=false,
   showspaces=false,
   tabsize=2,
   breaklines=true,
   showtabs=false,
   captionpos=b,frame=tb,
   aboveskip=3mm,
   belowskip=3mm,
   frame=tbrl, %t: top, r, b, l 
   frameround=tttt, 
   showstringspaces=false,
   captionpos=b,                    % sets the caption-position to bottom
   columns=flexible,
   basicstyle={\small\ttfamily},
   numbers=left,
   numberstyle=\tiny\color{gray},
   numberblanklines=false,
   breaklines=true,
   breakatwhitespace=true,
   xleftmargin=20pt, 
   xrightmargin=10pt, 
   framexleftmargin=15pt, 
   framexrightmargin=5pt,
   xleftmargin=20pt
}}
  {}

%Anpassungen für das Abkürzungsverzeichnis
\newglossarystyle{dottedlocations}{%
	\glossarystyle{list}%
	\renewcommand*{\glossaryentryfield}[5]{%
		\item[\glsentryitem{##1}\glstarget{##1}{##2}] \emph{##3}%
		\unskip\leaders\hbox to 2.9mm{\hss.}\hfill##5}%
	\renewcommand*{\glsgroupskip}{}%
}