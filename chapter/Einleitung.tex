\chapter{Einleitung}

Heutzutage erwarten die Internetnutzer kurze Ladezeiten, flüssige Bedienung und ständige Verfügbarkeit von den Webanwendungen, die zur Verfügung stehen. Dabei sind viele davon nicht in der Fähigkeit, mit rasant steigender Anzahl von Anfragen und großer Datenmenge effizient umzugehen. Die Anzahl von Anfragen und auch die Datenmenge, die von einer Webanwendung bedient werden können, sind nicht unbegrenzt, sondern abhängig von zahlreichen äußeren und inneren Faktoren. 

\section{Motivation und Ziel der Arbeit}

Die folgende Abschlussarbeit \textit{`\themaT`} befasst sich mit aktuellen Technologien (nasvatj mojet kakie? AngularJS 2, Spring MVC, NoSQL: MongoDB...), die ermöglichen, so eine Webanwendung zu implementieren, die mit einer plötzlich rasant steigenden Anzahl von Anfragen problemlos umgeht und weiterhin effizient funktioniert.

Für solche Webanwendungen wird eine Architektur aufgestellt. Diese Architektur ermöglicht den Entwicklern nicht nur skalierbare, sondern auch wartbare Webanwendungen zu implementieren. 

%\section{Aufbau der Arbeit}
%Erst Grundlagen/Konzepte, die relevant sind, um eine skalierbare und wartbare Webapplikation implementieren zu können.

%\section{Codeimplementierung}

%Der sämtliche Source Code liegt in dem folgenden GitHub Repository:

%\url{https://github.com/vlxxxfa/vlxxxfa.github.io}


