%\chapter{Einleitung}
%\chapter{\colorbox{green}{Einleitung}}
\chapter{Einleitung}

Internetnutzer erwarten heutzutage von den Webanwendungen, dass diese kurze Ladezeiten, flüssige selbsterklärende Bedienung und ständige Verfügbarkeit aufweisen. Viele Webanwendungen sind nicht in der Lage, mit rasant steigender Anzahl von Anfragen und großen Datenmengen effizient umzugehen.

Dies ist für erfolgreiche Projekte, die rapide populär werden und ein exponentielles Anwenderwachstum erleben, ein ernsthaftes Problem. Um von dem Projekterfolg profitieren zu können und diesen auszubauen, ist es überlebenswichtig, den Wachstumsanforderungen gerecht zu werden. Die Hardware stellt gegenwärtig kein großes Problem mehr dar: Die Cloud-Services wie z. B. \textit{Amazon} oder \textit{Microsoft Azure} ermöglichen den Zugang zu den fast unbegrenzten Hardwareressourcen. Die Herausforderung für Entwickler besteht darin, die Webanwendung so zu bauen, dass diese von dem Hardwareangebot Gebrauch machen kann. Die Wartungs- und Erweiterungsfähigkeit sind zwei weiteren wichtigen Punkte, die für den Erfolg unabdingbar sind. Um die Internetnutzer beizubehalten, sollen die neuen Features schnell implementiert und ausgerollt werden können, auch wenn das Projekt größer wird.

\section{Motivation und Ziel der Arbeit}
%\section{\colorbox{green}{Motivation und Ziel der Arbeit}}

Das Ziel dieser Arbeit ist, eine solche Architektur für Webanwendungen vorzustellen, die die Entwicklung von skalierbaren, wartungs- und erweiterungsfähigen Webanwendungen ermöglicht. Es wird des Weiteren ein Software-  und Frameworkstack vorgeschlagen, der diese Architektur abdeckt. Die vorgeschlagenen Software/Frameworks sind unter freien Lizenzen verfügbar.

Um die Realisierbarkeit und das Zusammenspiel aller Komponenten zu untersuchen, wird ein Prototyp implementiert und eigene Erfahrungen aus dem Entwicklungsprozess berichtet. Für den Prototyp wird der webbasierte Foto-Verwaltungs-Service gewählt. Diese Anwendung zeichnet sich dadurch aus, dass jeder Internetnutzer eigene Daten unabhängig von den anderen Nutzern verwalten kann, was bei vielen Webanwendungen der Fall ist. Anderenfalls sollen nicht nur triviale Textdaten verwaltet werden, sondern auch mehrere Dateien.
%\section{Aufbau der Arbeit}
%Erst Grundlagen/Konzepte, die relevant sind, um eine skalierbare und wartbare Webapplikation implementieren zu können.

%\section{Codeimplementierung}

%Der sämtliche Source Code liegt in dem folgenden GitHub Repository:

%\url{https://github.com/vlxxxfa/vlxxxfa.github.io}