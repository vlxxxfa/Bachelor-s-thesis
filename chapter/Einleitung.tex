%\chapter{Einleitung}
\chapter{\colorbox{green}{Einleitung}}

Heutzutage erwarten die Internetnutzer kurze Ladezeiten, flüssige Bedienung und ständige Verfügbarkeit von den Web-Applikationen, die zur Verfügung stehen. Dabei sind viele davon nicht in der Lage, mit rasant steigender Anzahl von Anfragen und großen Datenmengen effizient umzugehen.

Es ist ein ernsthaftes Problem für erfolgreiche Projekte, die rasant populär werden und ein exponentielles Anwenderwachstum erleben. Um von dem Projekterfolg zu profitieren und den auszubauen, es ist überlebenswichtig diesen Wachstumsanforderungen gerecht zu werden. Die Hardware ist heutzutage kein großes Problem mehr, die  Cloud-Services wie \textit{Amazon} oder \textit{Microsoft Azure} ermöglichen den Zugang zu den fast unbegrenzten Hardwareressourcen. Die Herausforderung besteht darin, die Web-Applikation so zu bauen, dass die von diesem Hardwareangebot Gebrauch machen kann. Die Wartung- und Erweiterungsfähigkeit sind zwei weitere wichtige Punkte, die für den Erfolg unabdingbar sind. Um die Internetnutzer zu halten, sollen die neuen Features schnell implementiert und ausgerollt werden können, auch wenn das Projekt größer wird.

%\section{Motivation und Ziel der Arbeit}
\section{\colorbox{green}{Motivation und Ziel der Arbeit}}

Das Ziel dieser Abschlussarbeit ist eine Architektur vorzuschlagen, die die Entwicklung der skalierbaren,  wartungs- und erweiterungsfähigen Web-Applikationen ermöglicht. Es wird weiterhin ein Software-  und Frameworkstack vorgeschlagen, der diese Architektur abdeckt. Die vorgeschlagenen Software/Frameworks sind unter freien Lizenzen verfügbar.

Um die Realisierbarkeit und das Zusammenspiel aller Komponenten zu untersuchen wird eine Prototyp implementiert und eigene Erfahrungen aus dem Entwicklungsprozess berichtet. Für den Prototyp wurde der webbasierte Foto-Verwaltungs-Service gewählt. Diese Applikation zeichnet sich dadurch aus, dass jeder Internetnutzer eigene Daten unabhängig von den anderen verwalten kann, was bei vielen Webanwendungen der Fall ist. Andererseits sollen nicht nur triviale Textdaten verwaltet werden, sondern auch mehrere Dateien.

%\section{Aufbau der Arbeit}
%Erst Grundlagen/Konzepte, die relevant sind, um eine skalierbare und wartbare Webapplikation implementieren zu können.

%\section{Codeimplementierung}

%Der sämtliche Source Code liegt in dem folgenden GitHub Repository:

%\url{https://github.com/vlxxxfa/vlxxxfa.github.io}


