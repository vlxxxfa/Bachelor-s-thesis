\chapter{Prototyp}

\begin{listingsboxShell}[label={lst:X}]{myshell}{Alle mongod-Prozesse zwingend stoppen}
> killall mongod
\end{listingsboxShell}


\section{Prototyp}
Kommentare zu den Fotos hinzufügen, Eingebettete Kommentare, siehe \url{http://ezproxy.bib.fh-muenchen.de:2125/doi/pdf/10.3139/9783446431225.014}



\section{Fazit}
Siehe Listing \ref{lst:X} \newline 
Doch wie der Begriff Not only SQL (NoSQL) andeutet, stehen beide Datenbanksysteme nicht unbedingt in direkter Konkurrenz zueinander, sondern können sich gegenseitig ergänzen. Dennoch, wenn es um die persistente Datenspeicherung bei Web-Anwendungen geht, stellen relationale Datenbanken nicht mehr die einzige Alternative dar. Bei eigenen Projekten wären Entwickler heute also gut beraten, die Vor- und Nachteile der beiden Systeme gegenüberzustellen und entsprechend den eigenen Anforderungen und Prioritäten zu bewerten. Muss das System mit großen Datenmengen effizient umgehen können? Werden hohe Anforderungen an Skalierbarkeit und Flexibilität der Datenbank gestellt? Sollen sich die Daten über mehrere Server verteilen lassen? Sind häufige Änderungen an der Datenstruktur in Zukunft zu erwarten? Wenn Sie die meisten dieser Fragen mit "Ja" beantworten, dann sollten Sie sich MongoDB zumindest näher anschauen.\newline\newline

Daten in MongoDB verfügen über ein flexibles Schema. Kollektionen (=Collections) erzwingt keine Struktur.



\section{Apache Cassandra}
Cassandra\footnote{Apache Cassandra: \url{http://cassandra.apache.org}, zugegriffen am 16. Dezember 2016} zählt, neben MongoDB\footnote{MongoDB: \url{https://www.mongodb.com}, zugegriffen am 16. Dezember 2016}, zu den derzeit populärsten NoSQL-Datenbanken. Cassandra war ursprünglich eine proprietäre Datenbank von Facebook und wurde 2008 als Open-Source-Datenbank veröffentlicht. Beispiele für weitere NoSQL-Datenbanken sind SimpleDB\footnote{SimpleDB: \url{https://aws.amazon.com/de/simpledb/}, zugegriffen am 16. Dezember 2016}, Google Big Table\footnote{Google Big Table: \url{https://research.google.com/archive/bigtable.html}, zugegriffen am 16. Dezember 2016}, Apache Hadoop\footnote{Apache Hadoop: \url{http://hadoop.apache.org}, zugegriffen am 16. Dezember 2016}, MapReduce\footnote{MapReduce: \url{http://hortonworks.com/apache/mapreduce/}, zugegriffen am 16. Dezember 2016}, MemcacheDB\footnote{MemcacheDB: \url{http://memcachedb.org}, zugegriffen am 16. Dezember 2016} und Voldemort\footnote{Voldemort: \url{http://www.project-voldemort.com/voldemort/}, zugegriffen am 16. Dezember 2016}. Unternehmen, die auf NoSQL setzen, sind unter anderem NetFlix\footnote{NetFlix: \url{https://www.netflix.com/de-en/}}, LinkedIn\footnote{LinkedIn: \url{https://www.linkedin.com/feed/}} und Twitter\footnote{Twitter: \url{https://twitter.com/?lang=en}}.\footnote{NoSQL: \url{http://www.searchenterprisesoftware.de/definition/NoSQL}, zugegriffen am 16. Dezember 2016}\newline

Cassandra ist als skalierbares, ausfallsicheres System für den Umgang mit großen Datenmengen auf verteilten Systemen (Clustern) konzipiert. Sie ist die beliebteste spaltenorientierte NoSQL-Datenbank und im Gegensatz zu MongoDB (C++) in Java geschrieben. Aufgrund ihrer architektonischen Eigenschaften wird Cassandra häufig in Big-Data-Projekten eingesetzt, kann in Zusammenarbeit mit einem Applikations-Server/Framework aber auch gut für komplexe Webanwendungen verwendet werden.
