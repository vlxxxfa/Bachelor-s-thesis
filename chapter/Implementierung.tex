%\chapter{Zu empfehlende Architektur}
%\chapter{Empfohlene Architektur}
\chapter{Implementierung}\label{implement}

\section{Fachliche Spezifikation}

\section{3-Schichten-Architektur}

\subsection{Präsentationsschicht (Frontend)}

Vorhanden ist, dass Frontend die Liste von Photos, dargestellt als List von bytesArrays (jeweils ein byteArray stellt ein Photo dar) bekommt und diese im Frontend anzeigt. Besser ist, nur die LIste von Photos-Ids zu bekommen und  Session in Frontend einschalten. Der User erhält dann nur einen Teil von Photos, falls gewünscht den weiteren Teil etc.

\subsection{Logikschicht (Backend)}

\subsection{Datenhaltungsschicht (Datenbank)}
%\subsection{MongoDB mit Java}
%\subsubsection{\colorbox{red}{MongoDB mit Java}}
%
%\begin{listingsboxJava}[label={lst:conn}]{myJava}{Verbindungsaufbau}
%public static void main(String[] args) {
%
%	MongoClient mongoClient = new MongoClient("localhost", 27017);
%        MongoDatabase db = mongoClient.getDatabase("test");
%        MongoCollection<Document> collectionOfZips = db.getCollection("zips");
%        
%        // weitere CRUD-Operationen mit der ausgewählten Kollektion
%}
%\end{listingsboxJava}
%
%\begin{listingsboxJava}[label={lst:X}]{myJava}{Skript zur Initialisierung der Replikationsgruppe}
%public static void main (String[] args) throws InterruptedException {
%        MongoClient client = new MongoClient(asList(
%                new ServerAddress("localhost", 27017),
%                new ServerAddress("localhost", 27018),
%                new ServerAddress("localhost", 27019)));
%                
%                // weitere CRUD-Operationen
%}
%\end{listingsboxJava}
%
%\begin{listingsboxShell}[label={lst:X}]{myshell}{Simulation des Server-Ausfalls 'PRIMARY'}
%m101:PRIMARY> rs.stepDown()
%
%Result:
%
%2016-12-19T21:24:12.739+0100 I NETWORK  [thread1]
%trying reconnect to 127.0.0.1:27018 (127.0.0.1) failed
%2016-12-19T21:24:12.760+0100 I NETWORK  [thread1]
%reconnect 127.0.0.1:27018 (127.0.0.1) ok
%m101:SECONDARY> 
%\end{listingsboxShell}
%
%Der aktuelle MongoDB Java Treiber ist in Version 3.4.0 verfügbar und kann bequem als Maven Dependency geladen werden.
% 
%\begin{listingsboxJava}[label={lst:mongoJDriver}]{myxml}{MongoDB Java Treiber als Maven Dependency, Version 3.4.0}
%<dependency>
%        <groupId>org.mongodb</groupId>
%        <artifactId>mongo-java-driver</artifactId>
%        <version>3.4.0</version>
%</dependency>
%\end{listingsboxJava}
%
%Um die Sicherung der Zugehörigkeit der Mitglieder zu konkreter Replikationsgruppe festzustellen, Zeilen 6-8...
%\begin{listingsboxJava}[label={lst:guarantee}]{myJava}{Sicherung der Zugehörigkeit zu konkreter Replikationsgruppe}
% public static void main (String[] args) throws InterruptedException {
%        MongoClient client = new MongoClient(asList(
%                new ServerAddress("localhost", 27017),
%                new ServerAddress("localhost", 27018),
%                new ServerAddress("localhost", 27019)), 
%                MongoClientOptions.builder()
%                        .requiredReplicaSetName("m101")
%                        .build());
%\end{listingsboxJava}



