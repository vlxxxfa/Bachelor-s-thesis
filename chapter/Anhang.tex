%\appendix
%\renewcommand\thesection{\Alph{section}}
%\addchap{Anhang}

%\clearpage
%\appendix
%\chapter{Anhang}
%\addcontentsline{toc}{chapter}{Anhang A}

\appendix
\renewcommand{\chaptermark}[1]{\markboth{Appendix\ \thechapter.\ #1}{}}

\chapter{Anhang}

\section{Fachliche Spezifikation}
Die folgende fachliche Spezifikation hat zum Ziel, dem Endbenutzer die Grundprinzipien der geplanten webbasierten skalierbaren Webanwendung für Foto-Verwaltungs-Service zu präsentieren.
Die skalierbare Software für den webbasierten Foto-Verwaltungs-Service ist dem Plan nach so zu implementieren, dass jeder Nutzer diese als eigene Fotoalben-Verwaltung nutzen kann, unabhängig von wachsenden Ansprüchen an die Leistungsfähigkeit der Anwendung.

Im Folgenden sind alle möglichen Szenarien dargestellt, die bei der Interaktion zwischen dem Besucher/Benutzer und des betrachteten Systems vorkommen könnten.

\begin{usecase}
\addtitle{Vorschau aller öffentlichen Fotoalben (=Preview of all public photo albums)}{}

\addfield{Kurzbeschreibung:}{Jeder Besucher kann alle vorhandenen öffentlichen Fotoalben sehen}
\addfield{Auslöser:}{Der Besucher verwendet den ihm bekannten Link für die Cloud-Fotoalben-Verwaltung}
\addfield{Eingabe:}{Den funktionierenden Link für die Cloud-Fotoalben-Verwaltung im Browser}
\addfield{Vorbedingung:}{Das System ist deployed}
\addfield{Ergebnis:}{Der Besucher landet auf die webbasierte Cloud-Fotoalben-Verwaltung und kann alle vorhandenen öffentlichen Fotoalben sehen}

\end{usecase}

\begin{usecase}
\addtitle{Ein Fotoalbum auswählen (= Show a photo album)}{}

\addfield{Kurzbeschreibung:}{Jeder Besucher kann den Inhalt jedes öffentlichen Fotoalbums ansehen}
\addfield{Auslöser:}{Der Besucher klickt auf das entsprechende Fotoalbum an, um seinen Inhalt ansehen zu können}
\addfield{Eingabe:}{\textbf{keine}}
\addfield{Vorbedingung:}{mind. ein Fotoalbum existiert}
\addfield{Ergebnis:}{Der Besucher kann den Inhalt des Fotoalbums sehen}

\end{usecase}

%\begin{usecase}
%\addtitle{Ein Foto im Vollformat ansehen (Show a photo in the full format)}{}
%
%\addfield{Kurzbeschreibung:}{Jeder Besucher kann jedes Foto im Vollformat sehen}
%\addfield{Auslöser:}{Der Besucher klickt auf das entsprechende Foto an, um es im Vollformat ansehen zu können}
%\addfield{Eingabe:}{\textbf{keine}}
%\addfield{Vorbedingung:}{mind. ein Foto existiert}
%\addfield{Ergebnis:}{Der Besucher kann im Vollformat das Foto ansehen}
%
%\end{usecase}

\begin{usecase}
\addtitle{Registrierung (=Register)}{}

\addfield{Kurzbeschreibung:}{Ein Besucher registriert sich als neuer Benutzer }
\addfield{Auslöser:}{Der potentielle Benutzer klickt auf den Button \textbf{'Register'} und legt ein neues Benutzerkonto selbst an}
\addfield{Eingabe:}{Benutzername, Passwort und E-Mail sind für die Registrierung unbedingt einzugeben}
\addfield{Vorbedingung:}{Der potentielle Benutzer ist unter dem eingegebenen Benutzername im System noch nicht registriert}
\addfield{Ergebnis:}{Das Benutzerkonto für den neuen Benutzer wird im System angelegt. Der Benutzer wird nach der \textbf{'Registrierung'} an die Startseite weitergeleitet}

\end{usecase}

\begin{usecase}
\addtitle{Anmeldung (=Login)}{}

\addfield{Kurzbeschreibung:}{Der registrierte Benutzer meldet sich im System an}
\addfield{Auslöser:}{Für die Anmeldung klickt der Benutzer auf den Button \textbf{'Please login'} und meldet sich mit seinem Benutzername und Passwort im System an}
\addfield{Eingabe:}{Benutzername und Passwort}
\addfield{Vorbedingung:}{Der Benutzer ist im System schon registriert}
\addfield{Ergebnis:}{Der Benutzer wird nach der erfolgreichen Anmeldung an eigene Startseite weitergeleitet}

\end{usecase}

\begin{usecase}
\addtitle{Fotoalbum anlegen (=Create a new photo album)}{}

\addfield{Kurzbeschreibung:}{Der angemeldete Benutzer legt sein neues Fotoalbum an}
\addfield{Auslöser:}{Für die Erzeugung eines neuen Fotoalbums klickt der Benutzer auf den Button \textbf{'Create a new photo album'}}
\addfield{Eingabe:}{Bezeichnung, Beschreibung und Abgrenzungsoption (= privat oder öffentlich) für sein zukünftiges Fotoalbum}
\addfield{Vorbedingung:}{Das Fotoalbum mit so einem Namen existiert bei dem angemeldeten Benutzer \textbf{nicht}}
\addfield{Ergebnis:}{Das Fotoalbum wird erzeugt und die Möglichkeit für das \textbf{'Select a photo for upload'} wird gleich freigeschaltet}

\end{usecase}

\begin{usecase}
\addtitle{Foto hochladen (=Upload a photo)}{}

\addfield{Kurzbeschreibung:}{Ein Benutzer lädt Fotos in ein existierendes Fotoalbum hoch}
\addfield{Auslöser:}{Für das Hochladen von Fotos in ein Fotoalbum klickt der Benutzer auf den Button \textbf{'Select a photo for upload'} und wählt ein Foto zum Hochladen aus}
\addfield{Eingabe:}{Gewünschtes Foto zum Hochladen auswählen}
\addfield{Vorbedingung:}{mind. ein Fotoalbum existiert}
\addfield{Ergebnis:}{Das ausgewählte Foto wird hochgeladen und die Möglichkeit für das \textbf{'Play a slideshow'} wird mit dem 1. hochgeladenen Foto gleich freigeschaltet}

\end{usecase}

\begin{usecase}
\addtitle{Foto löschen (=Delete a photo)}{}

\addfield{Kurzbeschreibung:}{Nur ein registrierter Benutzer kann die eigenen Fotos aus den Fotoalben löschen}
\addfield{Auslöser:}{Für das Löschen von Fotos in einem Fotoalbum wählt der Benutzer bestimmte Fotos aus einem Fotoalbum aus und klickt auf den Button \textbf{'Delete n photos'}. \textbf{n} steht für die Anzahl von Fotos}
\addfield{Eingabe:}{\textbf{keine}}
\addfield{Vorbedingung:}{Die löschenden Fotos existieren}
\addfield{Ergebnis:}{Die markierten Fotos sind gelöscht und sind in dem Fotoalbum nicht mehr vorhanden}

\end{usecase}

\begin{usecase}
\addtitle{Fotoalbum löschen (=Delete a photo album)}{}

\addfield{Kurzbeschreibung:}{Nur ein registrierter Benutzer kann die eigenen oder für ihn sichtbaren Fotoalben mit dem ganzen Inhalt löschen}
\addfield{Auslöser:}{Für das Löschen von Fotoalben klickt der Benutzer auf den Button \textbf{'Delete a photo album'}}
\addfield{Eingabe:}{\textbf{keine}}
\addfield{Vorbedingung:}{Das löschende Fotoalbum existiert}
\addfield{Ergebnis:}{Das Fotoalbum ist gelöscht und ist im System nicht mehr vorhanden}

\end{usecase}

%\subsection{Foto-Diashow abspielen - \textbf{optional}}
%\begin{usecase}
%\addtitle{Foto-Diashow abspielen (=Play a slideshow)}{}
%
%\addfield{Kurzbeschreibung:}{Ein Benutzer spielt Foto-Diashow ab}
%\addfield{Auslöser:}{Für das Abspielen von Foto-Diashow klickt der Benutzer auf den Button \textbf{'Play a slideshow'}}
%\addfield{Eingabe:}{\textbf{keine}}
%\addfield{Vorbedingung:}{mind. ein Foto ist in einem Fotoalbum vorhanden}
%\addfield{Ergebnis:}{Der Benutzer kann Foto-Diashow abspielen}
%
%\end{usecase}
%
%\subsection{Fotoalbum freigeben - \textbf{optional}}
%\begin{usecase}
%\addtitle{Fotoalbum freigeben (=Share a photo album)}{}
%
%\addfield{Kurzbeschreibung:}{Ein Benutzer lädt seine Bekannte/Freunde/Verwandte zum einem bestimmten Fotoalbum per E-Mail ein, um dieses verwalten zu können}
%\addfield{Auslöser:}{Für die Freigabe eines Fotoalbums klickt der Benutzer auf den Button \textbf{'Share a photo album'} und teilt per E-Mail über die Freigabe des Fotoalbums an die Gewünschten mit}
%\addfield{Eingabe:}{Die E-Mail-Adressen von Bekannten/Freunden/Verwandten für die Freigabe}
%\addfield{Vorbedingung:}{Der Eingeladene befindet sich nicht in der Liste der Freigegebenen}
%\addfield{Ergebnis:}{Der Eingeladene kann ein freigegebenes Fotoalbum verwalten}
%
%\end{usecase}
%
%\subsection{Fotoalbum als Gast verwalten - \textbf{optional}}
%\begin{usecase}
%\addtitle{Fotoalbum als Gast verwalten (=Manage a photo album as guest)}{}
%
%\addfield{Kurzbeschreibung:}{Ein Gast kann die für ihn freigegebenen Fotoalben verwalten}
%\addfield{Auslöser:}{Eine Einladung per E-Mail, die eine Freigabe in Form eines Links verfügt}
%\addfield{Eingabe:}{\textbf{keine}}
%\addfield{Vorbedingung:}{Eine Freigabe per E-Mail}
%\addfield{Ergebnis:}{Der eingeladene Gast kann das für ihn freigegebene Fotoalbum verwalten}
%
%\end{usecase}
%
%\subsection{Foto kommentieren - \textbf{optional}}
%\begin{usecase}
%\addtitle{Foto kommentieren (=Comment on a photo)}{}
%
%\addfield{Kurzbeschreibung:}{Nur ein autorisierter Benutzer kann Fotos kommentieren}
%\addfield{Auslöser:}{Für das Kommentieren von Fotos gibt der autorisierter Benutzer auf den Button \textbf{'Comment'} und gibt seinen Kommentar ein}
%\addfield{Eingabe:}{Seine Meinung zu dem gewählten Foto}
%\addfield{Vorbedingung:}{mind. ein Foto ist in einem Fotoalbum vorhanden und der Benutzer darf nicht unauthentifiziert sein}
%\addfield{Ergebnis:}{Das Foto verfügt über einen \textbf{neuen} Kommentar}
%
%\end{usecase}

%\subsection{}
%\begin{usecase}
%\addtitle{}{}
%
%\addfield{Kurzbeschreibung:}{}
%\addfield{Auslöser:}{}
%\addfield{Eingabe:}{}
%\addfield{Vorbedingung:}{}
%\addfield{Ergebnis:}{}
%
%\end{usecase}

%\section{Technische Spezifikation für die Entwickler}

%Eine technische Spezifikation dient dem Entwickler zur Orientierung, wie die geplante Software strukturiert ist und wie die in der Abbildung XXX dargestellten Module zusammenspielen, welche Frameworks, Programmiersprachen für Frontend und Backend genutzt werden.

\section{Skript zur Erstellung einer Replikationsgruppe}

\begin{listingsboxJavaScript}[label={lst:createReplSet}]{myJS}{Skript erstellt eine Replikationsgruppe inklusive 3 Server}
#!/usr/bin/env bash

mkdir -p replSet/data/db/rs1 replSet/data/db/rs2 replSet/data/db/rs3 logFiles/

mongod --replSet replSet --logpath "logFiles/1.log" --dbpath replSet/data/db/rs1
--port 27017 --oplogSize 64 --fork --smallfiles
mongod --replSet replSet --logpath "logFiles/2.log" --dbpath replSet/data/db/rs2
--port 27018 --oplogSize 64 --smallfiles --fork
mongod --replSet replSet --logpath "logFiles/3.log" --dbpath replSet/data/db/rs3
--port 27019 --oplogSize 64 --smallfiles --fork

sleep 5
#connect to one server and initiate the set
echo "Configuring a replica set"
mongo --port 27017 << 'EOF'

config = { _id: "replSet", members:[
              { _id : 0, host : "localhost:27017"},
              { _id : 1, host : "localhost:27018"},
              { _id : 2, host : "localhost:27019"} ]
};
rs.initiate(config);
EOF
\end{listingsboxJavaScript}

\section{Skript zur Erstellung von Shards, mit je einer Replikationsgruppe}

\begin{listingsboxJavaScript}[label={lst:createShards}]{myJS}{Skript erzeugt drei Shards, mit je einer Replikationsgruppe inklusive 3 Server, Teil 1}
# sharded_env.sh -> run in Terminal as: bash < sharded_env.sh

# script to start a sharded environment on localhost

# clean everything up
echo "killing mongod and mongos"
sudo killall mongod
sudo killall mongos
echo "removing data files"
rm -rf sharding/data/db/config
rm -rf sharding/data/db/shard*
rm -rf sharding/data/db/logFiles

# start a replica set and tell it that it will be shard0
echo "starting servers for shard 0"
mkdir -p sharding/data/db/shard0/rs0 sharding/data/db/shard0/rs1 
sharding/data/db/shard0/rs2 logFiles/

mongod --replSet s0 --logpath "logFiles/s0-r0.log" 
--dbpath sharding/data/db/shard0/rs0 --port 27017 --fork --shardsvr --smallfiles
mongod --replSet s0 --logpath "logFiles/s0-r1.log" 
--dbpath sharding/data/db/shard0/rs1 --port 27018 --fork --shardsvr --smallfiles
mongod --replSet s0 --logpath "logFiles/s0-r2.log" 
--dbpath sharding/data/db/shard0/rs2 --port 27019 --fork --shardsvr --smallfiles

sleep 5
# connect to one server and initiate the set
echo "Configuring s0 replica set"
mongo --port 27017 << 'EOF'
config = { _id: "s0", members:[
          { _id : 0, host : "localhost:27017" },
          { _id : 1, host : "localhost:27018" },
          { _id : 2, host : "localhost:27019" }]};
rs.initiate(config)
EOF
\end{listingsboxJavaScript}

\begin{listingsboxJavaScript}[label={lst:initReplica}]{myJS}{Skript erzeugt drei Shards, mit je einer Replikationsgruppe inklusive 3 Server, Teil 2}

# start a replicate set and tell it that it will be a shard1
echo "starting servers for shard 1"
mkdir -p sharding/data/db/shard1/rs0 sharding/data/db/shard1/rs1 
sharding/data/db/shard1/rs2 logFiles/

mongod --replSet s1 --logpath "logFiles/s1-r0.log" 
--dbpath sharding/data/db/shard1/rs0 --port 37017 --fork --shardsvr --smallfiles
mongod --replSet s1 --logpath "logFiles/s1-r1.log" 
--dbpath sharding/data/db/shard1/rs1 --port 37018 --fork --shardsvr --smallfiles
mongod --replSet s1 --logpath "logFiles/s1-r2.log" 
--dbpath sharding/data/db/shard1/rs2 --port 37019 --fork --shardsvr --smallfiles

sleep 5

echo "Configuring s1 replica set"
mongo --port 37017 << 'EOF'
config = { _id: "s1", members:[
          { _id : 0, host : "localhost:37017" },
          { _id : 1, host : "localhost:37018" },
          { _id : 2, host : "localhost:37019" }]};
rs.initiate(config)
EOF

# start a replicate set and tell it that it will be a shard2
echo "starting servers for shard 2"
mkdir -p sharding/data/db/shard2/rs0 sharding/data/db/shard2/rs1 
sharding/data/db/shard2/rs2 logFiles/

mongod --replSet s2 --logpath "logFiles/s2-r0.log" 
--dbpath sharding/data/db/shard2/rs0 --port 47017 --fork --shardsvr --smallfiles
mongod --replSet s2 --logpath "logFiles/s2-r1.log" 
--dbpath sharding/data/db/shard2/rs1 --port 47018 --fork --shardsvr --smallfiles
mongod --replSet s2 --logpath "logFiles/s2-r2.log" 
--dbpath sharding/data/db/shard2/rs2 --port 47019 --fork --shardsvr --smallfiles

sleep 5

echo "Configuring s2 replica set"
mongo --port 47017 << 'EOF'
config = { _id: "s2", members:[
          { _id : 0, host : "localhost:47017" },
          { _id : 1, host : "localhost:47018" },
          { _id : 2, host : "localhost:47019" }]};
rs.initiate(config)
EOF
\end{listingsboxJavaScript}


\begin{listingsboxJavaScript}[label={lst:initReplica}]{myJS}{Skript erzeugt drei Shards, mit je einer Replikationsgruppe inklusive 3 Server, Teil 3}
# now start 3 config servers
echo "Starting config servers"
mkdir -p sharding//data/db/config/config-a sharding//data/db/config/config-b 
sharding//data/db/config/config-c logFiles/

mongod --logpath "logFiles/cfg-a.log" --dbpath sharding/data/db/config/config-a
 --port 47040 --fork --configsvr --smallfiles
mongod --logpath "logFiles/cfg-b.log" --dbpath sharding/data/db/config/config-b 
--port 47041 --fork --configsvr --smallfiles
mongod --logpath "logFiles/cfg-c.log" --dbpath sharding/data/db/config/config-c 
--port 47042 --fork --configsvr --smallfiles

# now start the mongos on a standard port
mongos --logpath "logFiles/mongos-1.log" 
--configdb localhost:47040,localhost:47041,localhost:47042 --fork
echo "Waiting 60 seconds for the replica sets to fully come online"
sleep 60
echo "Connnecting to mongos and enabling sharding"

# add shards and enable sharding on the test db
mongo <<'EOF'
db.adminCommand( { addshard : "s0/"+"localhost:27017" } );
db.adminCommand( { addshard : "s1/"+"localhost:37017" } );
db.adminCommand( { addshard : "s2/"+"localhost:47017" } );
db.adminCommand({enableSharding: "qwertz"})
db.adminCommand({shardCollection: "qwertz.users", key: {_id:1}});
EOF
\end{listingsboxJavaScript}


